This thesis started from a simple neuronal model and coupled it with a mechanical system to reach stable oscillation. This goal led to the definition of multiple sensory feedback to couple the mechanical system with the control neurons. Analysis revealed the spike-like mixed feedback controller to be a reliable feedback source. Next, coupling two motor neuron allowed the system to reach symmetrical oscillation with greater amplitude. The final step was to allow a control on this amplitude oscillation with great energy efficiency. 

But, in my opinion, this thesis only opens the subject of control and many question can still be answered. Firstly, an interesting addition to this work would be to interconnect multiple pendulum by their controllers to try generate different gaits patterns between pendulum. Another step after reaching those gaits would be to try to switch dynamically between them in a smooth way to avoid unnatural and abrupt transitions.

On another point, the controller proposed here relies heavily on a very strong sensory feedback. An interesting challenge would be to use a very weak feedback but try to use neuromodulation to adapt the natural frequency of the HCO to match the frequency of the pendulum. Keeping the already existing neuromodulation for amplitude control in this system would also be very interesting.

Also, the performances of the controller are only discussed in comparison with itself and some natural control criteria. It would be useful to compare this controller to more classical approaches such as the PID. Natural comparison would be the tracking of the reference but also the amount of energy or force used. This would solidify the usefulness of this controller.
