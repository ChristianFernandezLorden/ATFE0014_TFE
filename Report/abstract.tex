\thispagestyle{plain}
\begin{center}
    \Large
    \textbf{\mytitle}
        
    \vspace{0.4cm}
    \large
    \mysubtitle
        
    \vspace{0.4cm}
    \textbf{\myname}
       
    \vspace{0.9cm}
    \textbf{Abstract}
\end{center}
The control of robotic locomotion poses important challenges. In particular, we are still very far from achieving in robotic locomotion control with the same degree of robustness and adaptability to unexpected environmental perturbations exhibited by moving biological systems.

In my master thesis research, I use biological neuron models to create artificial CPG to control a simple mechanical system in a robust and efficient manner. Akin to \citet{CPGRobot}, my inspiration is the known electrophysiology, sensory response, and modulation of biological CPGs \citet{crayfish,stickInsect,MARDER}.

My research started by exploring how a single neuron is able to robustly generate a high-amplitude periodic motion in a simple resonant mechanical system (a pendulum) without fine tuning of the neuron parameters and with weak sensory feedback and actuation. I found that only if the motor neuron exhibits a robust type of bursting \citet{burstingSlowFeedback,Franci2} it is able to robustly and rapidly adapt its excitable behavior to the unknown mechanical system's properties (damping, resonant frequency, mass, etc.).

I am now exploring how to use a pair of bursting neurons in a push-pull configuration to simultaneously achieve robust amplitude and oscillation frequency control and how to create network of bursting neuron to achieve robust and adaptable spatio-temporal coordination of coupled mechanical systems, akin to multi-legged 
locomotion.

The hope is that a good and robust control over simple mechanical systems is the first step toward efficient adaptive walking.

